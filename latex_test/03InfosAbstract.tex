%\chapter*{Allgemeine Informationen}
\thispagestyle{empty}
\pagestyle{empty}

%\begin{tabular}{p{0.3\textwidth}p{0.65\textwidth}}
%
%Vor- und Zuname: & \Author \\*[0.2cm]
%Institution: & Fachhochschule Salzburg GmbH \\*[0.2cm]
%Studiengang: & Informationstechnik \& System-Management \\*[0.2cm]
%Titel der Masterarbeit: & \Title \\*[0.2cm]
%Schlagwörter: & \Keywords  \\*[0.2cm]
%Betreuer an der FH: & \Advisor
%
%\end{tabular}

\section*{\Large\bfseries Kurzzusammenfassung}

In der heutigen Zeit ist eine funktionierende IT Infrastruktur für den organisierten Betrieb einer Schule unerlässlich. Diese Arbeit umfasst die Erneuerung der IT Infrastruktur des Royal Institute for Tourism and Hospitality (RITH) in Thimphu/Bhutan. Da sich der IT Sektor von Bhutan noch am Anfang der Entwicklung befindet, werden zu Beginn dieser Arbeit mehrere Lebenszyklusmodelle erläutert. Weiters werden für eine IT Infrastruktur relevante IT Kennzahlen erklärt. 

Ausgehend von der am RITH bereits vorhandenen Hardware und den Anforderungen an die IT Infrastruktur wird ein neuer IT Infrastruktur Plan erstellt. Wichtige Faktoren bei der Auswahl der neuen Hardware sind die Kosten, die Support Verfügbarkeit für Bhutan, Größe und Gewicht der Hardware sowie eine Ausgeglichenheit zwischen Funktionalität und Komplexität der Wartung. Der IT Infrastrukturplan wird im Detail erklärt und begründet. Eine genaue Erläuterung der Netzwerktechnik befindet sich im hinteren Teil dieser Arbeit. Dabei wird auf das IP Design, die Redundanz im Netzwerk und die Netzwerkkonfiguration des Betriebssystems vSphere eingegangen. 

Den Abschluss bildet eine Zusammenfassung und ein Ausblick auf die zukünftige IT Infrastruktur. Dieser wird durch Erweiterungsvorschläge abgerundet.

\section*{\Large\bfseries Abstract}

A working and up-to-date IT infrastructure for a school is one of the most essential things these days. This thesis deals with the renewal of the IT infrastructure of the Royal Institute for Tourism and Hospitality (RITH) located in Thimphu/Bhutan. Corresponding to the fact, that the IT sector of Bhutan is still at an early stage of development, lifecycle models will be discussed. Moreover, import characteristic numbers regarding IT infrastructures are adressed. 

Based on the existing hardware at RITH and the requirements to the IT infrastructure, a new IT infrascture plan will be created. Crucial factors regarding the selection of new hardware components were the costs, support availability for Bhutan, size and weight of the component and a good balance between functionality of the IT infrastructure and the difficulty of maintenance. The IT infrastructure plan will be explained in detail. Exact details about the network configuration are also described within this thesis. \\
The last part consists of a summary and suggestions regarding the future enhancement of the IT infrastructure. 