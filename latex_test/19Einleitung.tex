\chapter{Introduction}

\thispagestyle{standard}
\pagestyle{standard}

\section{Overview}

A working and up-to-date IT infrastructure for a school is one of the most essential things these days. This practical training deals with the renewal of the IT infrastructure of the \ac{RITH} located in Thimphu/Bhutan. Moreover a school management system had to be introduced.

Based on the existing hardware at \ac{RITH} and the requirements to the IT infrastructure, a new IT infrastructure has been created. Crucial factors regarding the selection of the new hardware components were the costs, support availability for Bhutan, size and weight of the component and a good balance between functionality of the IT infrastructure and the difficulty of maintenance. The IT infrastructure plan will be explained in detail. Exact configurations of the network infrastructure are also described within this report.

The new IT infrastructure plan has been implemented at \ac{RITH} and the new hardware installed. Tuition has been given to the IT administrators at \ac{RITH} about the operation of the IT infrastructure and about new technologies, that haven't been known yet. Moreover important documentation has been created regarding the operation and maintenance of specific IT infrastructure components.

This practical training deals with a part of a project that was founded in 2008 by a consortium between the Tourism Schools Salzburg, the degree course IMT at Salzburg University of Applied Sciences and the Foundation Urstein. The funding of the project was supported by the \ac{ADA}. The goal of this sub-project is the renewal of the IT infrastructure at the \ac{RITH} in Thimphu/Bhutan.
